\chapter{Non perturbative Physics}
\section{Introduction}
The course will mostly about non-perturbative methods or phenomena, very rarely you have general methods though. Every problem in NPP is a new problem, as it is very difficult to talk about general concepts. We will just do a lot of examples for non-perturbative problems.
\section{Motivating problems in QM}
\subsection{Theorem}
Want to proove the theorem that the ground-state of a QM system is non-degenerate. Consider therefore the following $N$-particle Schrödinger system
\be 
\left[-\sum_i \frac{\partial^2}{\partial x^2_i} + V(\{x_i\} ) \right] \Psi_n(x_1,x_2,\dots,x_N) = E_n \Psi_n (x_1, \dots, x_N).
\ee 
Where we assumed the system to be Hamiltonian, the wave-functions to be decaying on the boundary of $\mR^N$. How do we prove that the ground-state is unique.\\
Pick an arbitrary function $\Psi(x_1 \dots, x_N)$ and define the average energy
\bse 
\expval{E}= \int \md^N x \Psi \hat{H} \Psi=\int \md^N x \left[\half (\partial_x \Psi)^2 + V \Psi^2 \right]+ \lambda ( \norm{\Psi}^2 -1 ),
\ese 
you can always pick solutions to be purely real or purely imaginary which is why we left the complex conjugate off of $\Psi$, as we chose real solutions. Varying this functional w.r.t. $\Psi$ will give you the ground-state, where you have to impose $\int \md^N x \Psi^2 =1$ (which we did via Lagrange multiplier).
\subsubsection{Prove a small Lemma for this}
\begin{mybox}{}
If $\Psi_0$ is a ground state, then $\Psi_0$ has a definite sign.
\end{mybox}
Assume it to change sign, then we can consider a solution
\bse 
	\Psi^\prime  = \abs{\Psi_0}
\ese 
where the derivative can have a different sign
\bse 
\partial_i \Psi^\prime = \pm \partial_i \abs{\Psi_0}.
\ese 
Then $\Psi^\prime$ has the same $\expval{E}$, but $\Psi^\prime$ can not solve the Schrödinger equation as it is not smooth. Varying $\Psi^\prime$, we can now however always lower $\expval{E}$, this is equivalent to the statement that $\Psi_0$ was not a ground state in the first place. The ground state (g.s.) has to have a definite sign.
\subsubsection{Continue with proof}
\begin{mybox}{}
	Ground state is unique.
\end{mybox}
Assume $\Psi_0$ and $\Psi^\prime_0$ both to be g.s.. Then
\bse 
\int \Psi_0 \Psi^\prime_0 \md^N X = 0\quad \int \Psi \Psi^\prime_0 \md^N x > 0,
\ese 
where the latter follows from choosing $\Psi_0$ and $\Psi^\prime_0$ to be positive, which can be assumed as the operators are hermitian.\footnote{This holds only for QM because it describes a finite volume. If you go towards QFT, you have to consider the infinite volume limit, where you have infinitely many particles. There, most states are exponentially close to the ground state (by Boltzmann decay of energies). Due to the infinite volume-limit, these all go down to the ground state such that you will have a degenerate ground state.}
\subsection{Particle on a circle} 
We have the classical Lagrangian 
\bse 
\mL = \frac{\dot{\vec{x}}^2}{2} = \frac{R^2 \dot{\phi}^2}{2} = \half \dot{\phi}^2
\ese 
for a normalized radius. We have the canonical momentum 
\bse 
p = \frac{\partial \mL}{\partial \dot{\phi}}= \dot{\phi}
\ese 
and Hamiltonian
\bse 
H = \half p^2.
\ese 
Upon quantization 
\bse 
p,\phi \rightarrow \hat{p},\hat{\phi}, \quad [\hat{\phi},\hat{p}]=i
\ese 
we have
\bse 
\hat{\phi} \psi(\phi) = \phi \psi(\phi), \quad \hat{p} \psi(\phi) = -i \partial_{\phi} \psi(\phi).
\ese 
We then have
\bse 
H = - \half \frac{\partial^2}{\partial \phi^2}.
\ese 
\subsubsection{Theorem and couple to magnetic field}
\begin{mybox}{Theorem}
	The spectrum is discrete via the periodicity.
	\be 
	E_n= \half n^2,\quad n\in \Z,\quad \psi(\phi+2 \pi) = \psi(\phi).
	\ee 
\end{mybox}
We can couple to a magnetic field by substituting
\bse 
p \rightarrow p-A = - i \frac{\partial}{\partial \phi} - A = -i \frac{\partial}{\partial \phi} - \frac{\theta}{2 \pi }
\ese 
where we substituted $\theta$ (an angle) as this is actually an example for a \emph{theta-term}, which can be found in many things in QFT.
We can get this from
\bse 
\mL = \half \dot{\phi}^2 + \frac{\theta}{2 \pi} \dot{\phi},\quad p = \dot{\phi} + \frac{\theta}{2 \pi}, 
\ese 
with energy spectrum
\bse 
H=\half ( p-\frac{\theta}{2\pi})^2 \quad \Rightarrow \quad E_n = \half (n-\frac{\theta}{2 \pi})^2 
\ese 
\todo{draws graph form his notes} 
Increasing theta from $0$ to $\pi$ lets you go from excited states to ground states, we see that the g.s. is double degenerate. The excited states are essentially related by parity. We find
\bse 
E^{\theta=0}_n = \half n^2,\quad E^{\theta=\pi}_n=\half (n-\half)^2, \quad E^{\theta=\pi}_0 = E^{\theta=\pi}_1.
\ese 
Thus the ground state is double degenerate. You can see the degeneracy via parity from
\bse 
n-\half \stackrel{!}{=} - (n^\prime -\half) \quad \Leftrightarrow \; n^\prime = 1-n.
\ese You typically do not have degeneracies. They are normally due to symmetries. It is very difficult to fine-tune the system such that a large part of the spectrum is degenerate if you do not have a symmetry underlying the system.
\begin{mybox}{}
	Always think of symmetries if you have degeneracies.
\end{mybox}
\subsubsection{Symmetries of this system}
We are talking about global symmetries here.  We define our representation space of the states via the global symmetry (e.g. rep of Poincare symmetry), these are the only important ones.\\
It is not only symmetry of the action, as, especially in the presence of a theta-term, it is sometimes more subtle to see all the symmetries of the system. For now we look at symmetries of the Hilbert space. We get a hint from the Lagrangian for $\theta=0$\footnote{Go over to general coordinates}
\bse 
\mL = \half \dot{q}^2.
\ese 
One symmetry by differentiation is 
\bse 
q(t) \rightarrow q(t) +\alpha, \quad \alpha \in \mR.
\ese 
Note however that $\alpha=2 \pi$ in this system as $q$ is an angle, i.e. $q ~ q + 2\pi$, which is basically a gauge symmetry of the system on a circle. The angle is not an observable, $\cos q$ or $e^{iq}$ are observable, $q$ is not. $\alpha$ really parametrizes a circle, as we also have that $\alpha ~ \alpha + 2\pi$. \\
We have another $\Z_2$ symmetry
\bse
q \rightarrow - q,
\ese 
which we call \emph{parity- or p-symmetry}. In the presence of a theta-term it is more subtle to see this symmetry, will see later in pathintegral language. Now define these symmetries on a Hilbert space via unitary operators, i.e. $\mO_\alpha$:
\bse 
\mO_\alpha \hat{q} \mO^\dagger_\alpha = \hat{q} + \alpha.
\ese 
The parity operator for $\theta=0$ is
\bse 
P \hat{q} P^\dagger = - \hat{q}.
\ese 
In terms of $\ket{n}$-eigenstates 
\bse 
P : \ket{n} \rightarrow \ket{-n}, \quad \mO_\alpha: \ket{n} \rightarrow e^{i \alpha n} \ket{n},
\ese 
where the exact form is
\bse 
P = \sum_n \ket{-n}\bra{n}, \quad \mO_\alpha = \sum_n e^{i\alpha n} \ket{n} \bra{n}.
\ese 
Now we consider the Hamiltonian in the eigenbasis
\bse 
H_\theta = \sum_n \half (n-\frac{\theta}{2 \pi})^2 \ket{n}\bra{n},
\ese 
where our symmetries hold as we find
\bse 
[\mO_\alpha, H_\theta] = 0 \quad [H_{\theta=0}, P] = 0.
\ese 
The spectrum is symmetric by a shift of $\theta \rightarrow \theta + 2 \pi$. This is non-trivial, as the \emph{whole} spectrum gets reshuffled
\bse 
E^{\theta=2 \pi}_n = \half (n-1)^2 \equiv \half {n^\prime}^2, \quad n^\prime = n-1.
\ese 
How does this work ?\\
We know further
\bse 
P \mO_\alpha P = \mO_{-\alpha},\quad O(2)-\text{group}.
\ese 
The task is to find an operator $\hat{P}$ such that the spectrum with non-vanishing but specific theta is invariant
\bse 
E^{\theta=\pi}_n = \half (n-\half)^2 \text{ which is $\mO_\alpha$-symmetric},
\ese 
and $\hat{P}^2=1$. We can construct this by observing that
\bse 
n-\half \rightarrow - (n-\half ) \; \Leftrightarrow \; n \rightarrow -n+1 \; \Rightarrow \; \ket{n} \rightarrow \ket{-n+1}.
\ese 
This operator is found to be
\bse 
\hat{P} = \sum_n \ket{-n+1}\bra{n} ,\quad \Rightarrow \quad [H^{\theta=\pi},\hat{P}]=0.
\ese 
Can we understand this symmetry further by looking at the algebra which $\hat{P}$ obeys
\bse 
\hat{P} \mO_\alpha \hat{P} =\sum_{n,n^\prime,n^{\prime \prime}} \ket{-n+1} \bra{n} e^{i \alpha n^\prime} \ket{n^\prime}\bra{n^\prime} \ket{-n^{\prime \prime} +1}\bra{n^{\prime \prime}} = e^{i \alpha} \mO_{-\alpha},
\ese 
which is a different algebra. We can never get rid of this phase by redefining operators. This is called a \emph{central extension} of $O(2)$, via the \emph{centre element} $I_\alpha = e^{i\alpha}$. Note that $\mO^2_\pi =\mI$, which has the algebra
\bse 
\hat{P} \mO_\pi \hat{P} = e^{i \pi} \mO_{-\pi}=- \mO_\pi \quad \Leftrightarrow\; \hat{P} \mO_\pi = - \mO_{\pi} \hat{P}.
\ese 
Let us assume that $\ket{\psi}$ is an eigenstate which is non-degenerate with energy, $H\ket{\psi} = E \ket{\psi}$. Then, what is the energy of a state $\ket{\psi^\prime}=\mO_\pi \ket{\psi}$ ?
\bse 
H \underbrace{\mO_{\pi} \ket{\psi}}_{\ket{\psi}} = \mO_{\pi} H \ket{\psi} = E \underbrace{\mO_{\pi} \ket{\psi}}_{\ket{\psi^\prime}} 
\ese 
which is an eigenstate with $E$ and also
\bse 
H \underbrace{\hat{P}\ket{\psi}}_{\ket{\psi^{\prime \prime}}} = E \underbrace{\hat{P} \ket{\psi}}_{\ket{\psi^{\prime \prime}}} 
\ese 
which is also an eigenstate with $E$.\\
Using a phase $\eta,\tilde{\eta}$ we know that 
\bse 
\mO_\pi \ket{\pi} = \eta \ket{\psi}, \quad \eta= \pm 1, \; \ket{\psi^{\prime \prime}} = \tilde{\eta} \ket{\psi}, \; \tilde{\eta}=\pm 1.
\ese 
We get a contradiction of our assumption that the state is non-degenerate by looking at 
\todo{look at notes}.
\subsection{T'Hooft Anomaly matching}
This is a first example of anomaly matching. We used an explicit algebra to derive this. We can do this more generally in the pathntegral language for the particle on a circle.
\bse 
Z = e^{- \beta \hat{H}} = \int \mD q e^{- \half \int_0^\beta \md t \left[\half \dot{q}^2 + i \frac{\theta}{2 \pi} \do{q}\right]},
\ese 
note that you dont have an $i$ inside the bracket for Minkowskian action. This theta-term is a topological theta term, which does not change its realness if you go to Euclidean space.
Note that the difference 
\bse 
S_{\theta=2\pi} - S_{\theta=0} = i \md  t \dot{q} = i (q(\beta) - q(0)) = i 2 \pi k,\quad k \in \Z
\ese 
so paths at $\theta = 2 \pi$ and $\theta=0$ are weighted by the same weight
\bse 
e^{S_{\theta=2 \pi} - S_{\theta=0}} = e^{i 2 \pi k} = 1 
\ese 
Again looking at symmetries, we have the invariance of the PI weight under the two symmetries
\bse 
\mO_\alpha: q \rightarrow q + \alpha, \quad P: q \rightarrow - q,
\ese 
where $P$ obviously is an invarince at $\theta=0$. What about $\theta=\pi$ ? We observe
\bse 
e^{i \pi k} \stackrel{\stackrel{q \rightarrow - q}{\rightarrow}}{e^{-i \pi k}= e^{i \pi k} }.
\ese 
\subsubsection{t'Hooft anomaly}
\begin{mybox}{t'Hooft anomaly}
	Let $G$ be a global symmetry. If $G$ cannot be promoted to a gauge symmetry, in a way that the action is local, we say $G$ has a \emph{t'Hooft anomaly}.\\
\\
	Consequence, if you cannot promote a good (all operators transform under it) global symmetry to a gauge symmetry, then this implies that the ground state is non-trivial.
\end{mybox}
Non-trivial ground state here means a ground state that is not unique and gapped.\footnote{Note that is highly non-trivial to understand what it means to gauge the $P$ symmetry, but the $U(1)$ symmetry $\mO_\alpha$ can be gauged.}
\\
\\
We will observe that parity invariance is lost upon gauging of $U(1)$ symmetry $\mO_\alpha$, such that there is a t'Hooft anomaly.

\subsection{Comments on  t'Hooft anomalies}



























\section{Lectures when I was gone}
\section{Spinors in $1+1$-dimension, an example of axial anomaly}
Write Dirac spinors as eigenstates of the Dirac oerator, where the bar indicates a left eigenstate\footnote{The reason we distinguish this is because $i\slashed{D}$ is not Euclidean in Minkowski space.}
\subsection{Fujikawa method}
Fujikawa's method is a way of deriving the chiral anomaly in quantum field theory.
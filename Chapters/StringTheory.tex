\chapter{String Theory}
Ttring theory is a two dimensional conformally invariant field
theory, but not a quantum field theory in four spacetime dimensions.
\section{Motivation}
\begin{enumerate}
	\item Modern physics is primarily built on three pillars that have held up to experiment time and again:
	\begin{enumerate} 
	\item \textbf{Special relativity} is the framework of choice when describing \emph{fast-moving} objects.
	\item  \textbf{General relativity} prevails in the face of objects so \emph{massive} that they bend spacetime itself.
	\item  \textbf{Quantum mechanics} claims to describe physics down to the \emph{smallest} level.
	\end{enumerate} 
	\item  But what if something is both small \emph{and} fast? To describe such systems, special relativity and
	quantum mechanics were beautifully incorporated into a multiparticle, relativistic framework called
	\textbf{quantum field theory} - the most successful physical theory yet, tested to excruciating precision.
	\item Or, what about systems that are massive yet small (and perhaps fast)? Clearly, for something to be
	both massive and small implies that we are looking at high densities on short length scales where
	gravity becomes important and of a magnitude comparable to the other forces. We are entering
	an exotic regime of physics involving systems such as the early universe and (rotating) black holes.
	Quantum field theory in its current state is of no use here, as it discards gravity from the outset. In
	fact, no field-theoretic description of gravity has been found that is both strictly local down to the
	smallest level and consistently quantizable, i.e. that results in a renormalizable quantum theory.
	\item As suggested by the Wilsonian interpretation of QFT, a fundamentally new (nonlocal) picture of
	the microscopic degrees of freedom is needed to make headway. This is where \textbf{string theory} comes
	in, whose central axiom is that \emph{the fundamental objects in Nature are one-dimensional rather than
	pointlike}. Combined with the standard kinematics of \textbf{general covariance}\footnote{General covariance is the paradigm that the form of physical laws should be invariant under arbitrary differentiable coordinate transformations. This statement is motivated by the conviction that coordinates do not exist in nature, and are only artificies of our description. Hence, which ones we choose should play no physical role.} and the usual procedure
	of \textbf{quantization}, this simple statement has resulted in an amazingly rich, mathematically intricate
	and conceptually insightful framework. In particular, string theory leads to a unified description of
	all forces, a divergence-free UV completion of QFT, and it recovers Einstein gravity at low energies.
\end{enumerate}
\subsection{Central properties}
\label{subsec:stringProperties}
\begin{enumerate}
	\item There is only \emph{one} free parameter in string theory, the \textbf{string length} $\ell_s$ which (due to current limits
	of high-energy colliders) can take any value in the range
	\bse 
	\text{(Planck length)}\quad 
	10^{ −35} m < \ell_s < 10^{ −19} m
	\quad \text{(TeV scale)}
	\ese 
	This is in stark contrast to GR and QFT where all masses and higher couplings are input parameters
	that have to be taken from experiment. String couplings are given by expectation values of a
	dynamical field, the scalar dilaton $\phi$. That means they can be calculated from within the theory!
	\item String states can be classified into two regimes:
	\begin{enumerate}
	\item  In the \textbf{low-energy limit} of distances much larger than $\ell_s$ , strings appear pointlike. Integrating
	out the massive string tower results in a low-energy effective theory of only the massless excita-
	tions. These are found to model gauge interactions and gravity. Conformal invariance of the field
	theory on the worldsheet requires that to lowest order, gravity obeys the Einstein equations.
	\item  The \textbf{ultraviolet regime} resides at distances of the order of $\ell_s$. The extended nature of the string
	becomes important, rendering the theory nonlocal with important consequences for interactions:
	Sharp vertices at which interactions are localized in space and time no longer exist.\footnote{In point-particle theories, the sharp localization of vertices is responsible for the appearance of divergent amplitudes.} Locally, the
	string always appears free with interactions encoded solely in the global worldsheet topology.
\end{enumerate}
	\item In string perturbation, each loop order (for a given process) contains only a single diagram. By
	contrast, the number of Feynman graphs in QFT grows factorially. Due to a feature called \textbf{duality},
	there is no need to sum over the scattering channels $s, t, u$. They are all the same in string theory.
	\item A string can be open or closed. Open strings generate Yang-Mills theory, closed strings produce
	gravity. Since open strings can close up and vice versa, gravity and Yang-Mills are dynamically
	related. One automatically implies existence of the other, as it must be, to allow for effects such as
	energy stored in an electric field to gravitate itself. This is what is meant with the statement that
	string theory provides a unified description of all forces.
\end{enumerate}


\section{Classical bosonic string}
\subsection{Bosonic string action}
\begin{mybox}{Nambu-Goto action }
	 The \textbf{Nambu-Goto action} of the classical bosonic string spanning the worldsheet $\Sigma$ is defined as
	\be 
	\label{eq:stringNGaction}
	S_{NG}[X]= - T \int_\Sigma \md A.
	\ee 
	$T$ is the string tension and $\md A=\sqrt{-\det(\vec{G})}\md \tau \md \sigma$ is the area element of $\Sigma$ with coordinates $\vec{\xi}=(\xi^0,\xi^1)=(\tau,\sigma)$. The components of the \textbf{pullback} $\vec{G}$ of the ambient space $\eta\munu$ onto $\Sigma$ are 
	\be 
	\label{eq:stringPullback}
	G_{ab}= \frac{\partial X^\mu}{\partial \xi^a} \frac{\partial X_\mu}{\partial \xi^b},\quad  a,b \in \{0,1\}, \; \mu \in \{0,\dots,d-1\}.
	\ee 
	\end{mybox} 
\begin{mybox}{Polyakov action}
	To eliminate the square root in \ref{eq:stringNGaction}, we introduce the \emph{worldsheet metric } $h^{ab}(\tau,\sigma)$ as an auxiliary (symmetric two-tensor) field and define the \emph{Polyakov action} 
	\be 
	\label{eq:stringPaction}
	S_P [X,h] = -\frac{T}{2} \int_\Sigma \md^2 \xi \sqrt{-h} h^{ab} G_{ab},\quad \text{with } h = \det(\vec{h}).
		\ee 
\end{mybox}
The bosonic string field $X^\mu(\tau,\sigma)$ in $G_{ab}$ provides an embedding of the worldsheet into ambient space. $X^\mu$ is a spacetime vector but a scalar on the worldsheet (due to the absence of worldsheet indices). Hence $S_P$ describes $d$ scalar fields $X^\mu$ coupled to the dynamical worldsheet metric $h_{ab}$. 
\begin{enumerate}
\item Since the spacetime coordinates $X^\mu$ of the string are promoted to dynamical fields, spacetime becomes a derived concept. The fundamental object is the field theory on the worldsheet.
\item $S_P$ and $S_{NG}$ are classically equivalent, i.e. upon enforcing the equation of motion for the auxiliary field $h_{ab}$. However, this equivalence does not extend to the quantum level.
\item \ref{eq:stringPaction} is the most general bosonic string action imaginable. $S_P$ could be modified in two ways
\begin{enumerate}
	\item A \emph{cosmological constant term} 
	\bse 
	S_\Lambda = \Lambda \int_\Sigma \md^2 \xi \sqrt{-h}
	\ese 
	could be added, but this would spoil Weyl invariance which will turn out to be vita for consistency of the CFT on the worldsheet.
	\item We might also include an \emph{Einstein-Hilbert term}
	\bse 
	S_{EH} = \frac{\lambda_{EH}}{4 \pi} \int_\Sigma \md^2 \xi \sqrt{-h}\mathcal{R} 
	\ese 
	with $\mathcal{R}$ the Ricci scalar of the worldsheet. But this is a total derivative and hence introduces no new dynamics (corresponding to the fact that two-dimensional gravity is dynamically trivial).
\end{enumerate}
\end{enumerate}
\subsection{Symmetries}
\subsubsection{Symmetries of the Polyakov action}
$S_P$ enjoys several symmetries, where it is important to distinguish between spacetime symmetries in $\mR^{1,\md-1}$ (taken to be flat) and worldsheet symmetries on $\Sigma$ (dynamic). $S_P$ is invariant under 
\begin{enumerate}
	\item $\md$-dimensional \emph{spacetime} \textbf{Poincaré transformations} 
	\bse 
	X^\mu \rightarrow \Lambda^\mu_\nu X^\nu+V^\mu,
	\ese
	with $\Lambda^\mu_\nu$ in the Lorentz group $SO(1,\md-1)$ and $V^\mu \in \mR^{1,\md-1}$ a translation. The associated conserved charges (according to Noether's theorem\ref{subsec:noethersymmetries}) are energy, momentum and angular momentum.
	\item Local \emph{worldsheet}  \textbf{diffeomorphisms} 
	\bse 
	\xi^a \rightarrow \xi^a + \epsilon^a (\xi)
	\ese 
	under which the string field $X^\mu$ transforms as
	\bse 
	\delta X^\mu= \epsilon^a\partial_a X^\mu,
	\ese
	 the metric $h_{ab}$ as 
	\bse 
	\delta h_{ab} = \nabla_a \epsilon_b +\nabla_b \epsilon_a,
	\ese 
	and the object $\sqrt{-h}$ as $\delta \sqrt{-h}= \partial_a (\epsilon_a \sqrt{-h})$, i.e. like a scalar density of weight $p=1$.
\item Local \textbf{Weyl transformations} 
\bse 
h_{ab} \rightarrow \Lambda(\vec{\xi}) h_{ab} 
\ese 
parametrized by $\Lambda(\vec{\xi})=e^{\omega(\vec{\xi})}$ with $\omega(\vec{\xi})\in \mR$ for convenient series expansion. This symmetry is special in that it arises only for two-dimensional world\emph{sheets} (as opposed to, say, membranes), making strings generalization of point-particles. This symmetries requires $T^a_a=0$ and is crucial for consistent string quantization. 
\end{enumerate}
\subsubsection{Killing symmetry}
The effect on the metric $h_{ab}$ of certain diffeomorphisms $\epsilon_a$ that fulfill the \text{conformal Killing equation} \ref{eq:ctkilling}
\bse 
P^c_{ab} \epsilon_c = (\vec{P} \vec{\epsilon})_{ab} = 0
\ese 
can be undone by a Weyl rescaling $\Lambda^{-1}$. The linear operator $\vec{P}$ is defined via
\be
\label{eq:stringConformalKilling} 
\delta h_{ab} = \nabla_a \epsilon_b + \nabla_b \epsilon_a = \underbrace{\nabla_a \epsilon_b+\nabla_b \epsilon_a - \nabla^c\epsilon_c h_{ab}}_{P^c_{ab} \epsilon_c} + \underbrace{\nabla^c \epsilon_c}_{\Lambda} h_{ab}.
\ee 
These $\epsilon_a$ are the \emph{conformal Killing vectors}. Every such $\epsilon_a$ yields a conserved current
\be 
J^a_\epsilon = T^{ab}\epsilon_b \quad \text{with} \quad \nabla_a J^a_\epsilon = 0 
\ee 
where the conservation follows from the tracelessness of the energy-momentum tensor. The number of such $\epsilon_a$ is infinite and hence infinitely many conserved currents arise.
\subsubsection{Energy-momentum tensor}
\begin{mybox}{Energy-momentum tensor} 
The energy-momentum tensor is defined as the variation of $S_P$ w.r.t. to the worldsheet metric
\be
\label{eq:stringEMtensor}
T_{ab} = \frac{4 \pi}{\sqrt{-h}} \frac{\delta S_P}{\delta h^{ab}} = - \frac{1}{\alpha^\prime} \left(G_{ab} - \half h_{ab} G^c_c\right).
\ee 
It is traceless $T^a_a=0$ (as a consequence of Weyl invariance), and (for on-shell $X^\mu$) constitutes the conserved current $\nabla^a T_{ab}=0$ w.r.t. local worldsheet diffeomorphisms.
\end{mybox}
The equation of motion 
\bse 
T_{ab}=0
\ese 
for $h_{ab}$ implies
\bse 
G_{ab} = \frac{G^c_c}{2} h_{ab},
\ese 
i.e. on-shell $h_{ab}$ is proportional to the pullback.

\subsection{Gauge-fixing}
On a $D$-dimensional worldmembrane, $h_{ab}$ has $\frac{D}{2}(D+1)$ degrees of freedom, while diffeomorphisms plus Weyl rescalings account for $(D+1)$ parameters. Precisely in $D=2$ do we have equally many transformational parameters as metric degrees of freedom. Two more features exclusive to $D=2$ are that the Riemann-tensor has only one degree of freedom given by the Ricci scalar $\mathcal{R}$
\be 
R_{abcd} = \frac{\mathcal{R}}{2} ( h_{ac} h_{bd} - h_{ad} h_{bc}),
\ee 
and second, that under Weyl rescaling $\Lambda(\vec{\xi}), \mathcal{R}$ transforms as
\bse 
\mathcal{R} \rightarrow \mathcal{R} - \vec{\nabla}^2\Lambda(\vec{\xi}). 
\ese 
Choosing $\Lambda(\vec{\xi})$ such that $\mathcal{R}= \vec{\nabla}^2 \Lambda(\vec{\xi})$ (locally, this is always possible) thus implies
\bse 
R_{abcd}=0 \quad \forall a,b,c,d.
\ese 
This means we can always transform the worldsheet so that locally, it resembles flat space. Once space is flat, we can transform coordinates, i.e. apply a diffeomorphism to bring the metric into Minkowskian shape $h_{ab} = \eta_{ab}$. This procedure of fixing the metric is called (partially) \emph{fixing the gauge}.
\begin{enumerate}
	\item It leaves a large \emph{residual gauge symmetry} generated by the conformal Killing vectors $\vec{\epsilon}$ mentioned above. Since these leave the metric invariant, they still represent an unphysical gauge symmetry in our description even after the metric has been fixed.
	\item Worldsheets may exhibit topological obstructions to fixing the metric globally. In this case there remain parameters in the metric, so-called \emph{moduli}, which cannot be removed by a conformal drescaling and diffeomorphisms. These moduli are the global properties of worldsheets that account for string interaction (mentioned in item $2$ of \ref{subsec:stringProperties}).
\end{enumerate}
\subsubsection{Flat gauge}
In \emph{flat gauge} $h_{ab}=\eta_{ab}$, the Polyakov action reduces to the action of $d$ \emph{free} scalar fields,
\be
\label{eq:stringPactionflatgauge}
S_P[X]= \frac{T}{2} \int_\Sigma \md^2 \xi \left[(\partial_\tau \vec{X})^2 - (\partial_\sigma \vec{X})^2\right].
\ee 
\subsubsection{Lightcone coordinates}
Lightcone coordinates $\xi^\pm = \tau \pm  \sigma$ are convenient, e.g. when treating closed string mode expansions with right- and left-moving modes $\vec{\alpha}^\pm_n$ ( $+$ right-moving, $-$ left-moving). The metric in \emph{lightcone gauge} reads
\bse 
h_{\pm \pm} =0, \; h_{\pm \mp} = - \half, \; \text{i.e. } \vec{h}= 
\begin{pmatrix}
	0& - \half \\
	-\half & 0 \\
\end{pmatrix},
\;
\vec{h}^{-1} = 
\begin{pmatrix}
	0 & -2 \\
	-2 & 0 \\
\end{pmatrix},
\ese 
yielding the line element 
\bse 
\md s^2 = h_{ab} \xi^a \xi^b = - \md \tau^2 + \md \sigma^2 = - \md\xi^+ \md \xi^-. 
\ese 
The Jacobian of the transformation $\begin{pmatrix}
\tau \\
\sigma \\
\end{pmatrix}
\rightarrow \begin{pmatrix}
\xi^+ \\
\xi^- \\
\end{pmatrix}
= 
\begin{pmatrix}
\tau + \sigma \\
\tau - \sigma \\
\end{pmatrix}
$ from worldsheet to lightcone coordinates has determinant
\bse 
\abs{\det(\vec{J})} = \abs{\det\begin{pmatrix}
		\partial_\tau \xi^+ & \partial_\sigma \xi^+ \\
		\partial_\tau \xi^- & \partial_\sigma \xi^- \\
\end{pmatrix}}
 = 
 \abs{\det \begin{pmatrix}
 		1 &1 \\
 		1 & -1 \\
 \end{pmatrix}}
=2.
\ese 
Thus the measure becomes 
\bse 
\md^2 \xi = \md \tau \md \sigma = \half \md \xi^+ \md \xi^-
\ese 
and the partial derivatives are $\partial_\pm = \half (\partial_\tau \pm \partial_\sigma)$.\\
\\
The Polyakov action and energy-momentum tensor in lightcone coordinates read 
\be
\label{eq:stringPactionEMtensorLightcone} 
S_P[X] = T \int_\Sigma \md^2 \xi \partial_+ \vec{X} \cdot \partial_- \vec{X},\quad T_{\pm \pm} = - \frac{1}{\alpha^\prime} \partial_\pm \vec{X} \cdot \partial_\pm \vec{X}.
\ee 
Tracelessness translates into $T_{\pm \mp}=0$, and conservation into $\partial_\mp T_{\pm \pm}=0 \Rightarrow T_{\pm \pm} (\xi^{\pm})$. It is important to remember that in flat gauge, the metric's equation of motion $T_{ab}=0$ still has to be enforced as constraint. Partially fulfilled already by tracelessness, this only amounts to $T_{\pm \pm}=0$.\\
The conformal Killing equation $(\vec{P}\vec{\epsilon})_{ab}=0$ in lightcone gauge, where now $\vec{\epsilon}=(\epsilon_+,\epsilon_-)$, becomes the statement $\partial_\pm \epsilon_\pm=0$. Using $\epsilon^\pm = h^{\pm a} \epsilon_a = h^{\pm \mp} \epsilon_\mp = -2 \epsilon_\mp$, this means
\bse 
\partial_\mp \epsilon^\pm = 0 \quad \Rightarrow \quad \epsilon^\pm = \epsilon^\pm(\xi^\pm),
\ese 
i.e. the $\epsilon^\pm$ are chiral.

\subsection{Mode expansion}
\subsubsection{Equation of motion and mode expansion}
Varying the Polyakov action \ref{eq:stringPaction} w.r.t. the bosonic string field yields the free wave equaton
\be 
\label{eq:stringEomBosonic}
(\partial^2_\tau - \partial^2_\sigma) X^\mu = 0 = \partial_+ \partial_- X^\mu 
\ee 
provided the boundary terms vanish. 
\begin{enumerate}
	\item he closed string has cancelling periodic boundaries.
	\item The open string requires \emph{Neumann} ($\partial_\sigma X^\mu =0$) and/or \emph{Dirichlet} ($\delta X^\mu = 0 = \partial_\tau X^\mu$) boundaries at both ends $\sigma \in \{0,l\}$.
\end{enumerate}
Each has a different mode expansion, e.g. the open NN string expansion is
\be
\label{eq:stringOpenModeExpansion}
X^\mu = x^\mu + \frac{p^\mu \tau}{T l} + i \sqrt{2 \alpha^\prime} \sum_{n\neq 0} \frac{\alpha^\mu_n}{n} e^{-i \frac{\pi}{l} n \tau} \cos(\frac{n\pi \sigma}{l}).
\ee 
\subsubsection{Virasoro algebra}
From 
\bse 
\{X^\mu (\tau,\sigma), \Pi^\nu (\tau,\sigma^\prime) \}_{PB} = \eta^{\mu \nu} \delta(\sigma-\sigma^\prime) \quad \text{with } \Pi^\mu = T\partial_\tau X^\mu,
\ese 
the \emph{Poisson bracket} forthe modes follows as
\bse 
\{\alpha^\mu_m,\alpha^\nu_n \}_{PB} = - i m \eta^{\mu \nu} \delta_{m,-n} 
\ese 
(for both left- and right-movers). Also, 
\bse 
\{x^\mu, p^\nu \}_{PB} = \eta^{\mu \nu}.
\ese
Inserting the $\partial_\pm X^\mu$ that result from \ref{eq:stringEomBosonic} into $T_{\pm \pm}$ from \ref{eq:stringPactionEMtensorLightcone} yields the mode expansion
\be 
\label{eq:stringModeExpansionVirasoro}
T_{\pm \pm} = 4 \alpha^\prime \sum_{m\in\Z} L^\pm_m e^{-i \frac{2\pi}{l} m \xi^\pm} 
\ee 
in terms of the \emph{Virasoro generators} $L_m$ (appeared also in \ref{subsubsec:virasoro}). The equation of motion (or constraint if $h_{ab}$ is fixed) $T_{ab}=0$ thus implies the \emph{Virasoro constraints} 
\be
 \label{eq:stringVirasoroConstraints}
L^\pm_m = 0 \quad \forall m \in \Z.
\ee 
In particular, the \emph{Hamiltonian}, which for the open string reads
\begin{align} 
\label{eq:stringHamiltonianOpenBosonic}
H_{op} &= \frac{\pi}{l} L_0 =\frac{\pi}{l} \half \sum_{n\in\Z} \vec{\alpha}_{-n} \cdot \vec{\alpha}_n \\
&= \frac{\pi}{l} \left(\half \vec{\alpha}^2_0 + \half \sum_{n\neq 0} \vec{\alpha}_{-n} \cdot \vec{\alpha}_n\right)= \frac{\pi}{l} \left(\alpha^\prime \vec{p}^2+ \sum_{n=1}^\infty \vec{\alpha}_{-n} \cdot \vec{\alpha}_n\right), \nonumber
\end{align}
must vanish due to $T_{ab}=0$ which implies the (classical open string) \emph{mass shell condition}
\be 
\label{eq:stringMassShellCondition}
M^2 =- \vec{p}^2 = \frac{1}{\alpha^\prime} \sum_{n=1}^\infty \vec{\alpha}_{-n} \cdot \vec{\alpha}_n = \frac{N}{\alpha^\prime}.
\ee 
For closed strings, 
\bse 
H_{cl} = \frac{2\pi}{l} (L^+_0 + L^-_0) \propto \partial_+ + \partial_- \propto \partial_\tau \stackrel{!}{=} 0
\ese 
implements time reparametrization invariance.




\section{Bosonic string quantization}
There are three popular ways to quantize string theory, each with its own merits and downsides.
\begin{enumerate}
	\item In the (old) \emph{canonical quantization}, the Virasoro constraints \ref{eq:stringVirasoroConstraints} are not implemented until we reach the quantum level. This manifestly retains the Lorentz covariance of the classical theory, but a unitary quantum theory is ensured only in a critical number of spacetime dimensions $d_{crit}$.
	\item \emph{Lightcone quantization} enforces Virasoro constrains already at the classical level, resulting in a manifestly unitary quantum theory, but Lorentz covariance holds only in $d = d_{crit}$.
	\item (Modern) \emph{path-integral quantization} uses the Faddeev-Popov gauge fixing procedure. Criticality becomes equivalent to closure of the BRST algebra, which can occur only in $d=d_{crit}$.
\end{enumerate}
\subsection{Canonical quantization}
\subsubsection{Canonical commutation relations}
Canonical quantization promotes all fields to operators and postulates the replacement $\{\cdot,\cdot \}_{PB}\rightarrow \frac{1}{i} [\cdot,\cdot]$, resulting in the canonical commutation relations

\begin{align*}
	[X^\mu(\tau,\sigma),\Pi^\nu (\tau,\sigma^\prime)] &= i \eta^{\mu \nu} \delta(\sigma-\sigma^\prime),\\
	[\alpha^\mu_m,\alpha^\nu_n] &= m \eta^{\mu \nu} \delta_{m,-n}, \\
	[x^\mu,p^\nu] &= i \eta{ \mu \nu}.
\end{align*}
Reality $X^\mu \in \mR$ at the classical level implies hermiticity $(X^\mu)^\dagger = X^\mu$ at the quantum level which in turn requires $(\alpha^\mu_m)^\dagger=\alpha^\mu_{-m}$. This carries over to the Virasoro generators $L^\dagger_m= L_{-m}$.

\subsubsection{Normal ordering}
As always, this procedure is terribly ambiguous because there is nothing to tell us the ’correct’ order within products of noncommuting operators. Hence we simply \emph{define} the \textbf{normal ordering} to be 
\be 
N(\alpha^\mu_m \alpha^\nu_n) = \left\{ \begin{array}{ll}
\alpha^\mu_m \alpha^\nu_n & \text{for } m\leq n,\\
\alpha^\nu_n \alpha^\mu_m & \text{for } n< m,\\
\end{array}\right\}
\ee 
and use this prescription to promote the Virasoro generators to the quantum theory as the operators
\be 
\label{eq:stringVirasorogenerators}
L_m = \half \sum_{n\in \Z} N({\vec{\alpha}}_{m-n} \cdot {\vec{\alpha}}_n).
\ee 
Actual ambiguity arises only in $L_0$ because modes $\alpha^\mu_m,\alpha^\nu_n$ are noncommuting only if $m=-n$ and for $m\neq 0$. By defining $L^{cl}_0 = L^{qu}_0 -a$\footnote{$L^{cl}_0$ and $L^{qu}_0$ are both quantum operators. The superscripts merely indicate that $L^{cl}_0$ has the structure of the classical Virasoro generators without normal-ordering prescription whereas $L^{qu}_0$ does, i.e. is precisely the one defined in \ref{eq:stringVirasorogenerators}.}, where $a$ follows from
\begin{align*}
	L^{cl}_0 &= \half \sum_{n\in\Z} \vec{\alpha}_{-n} \cdot \vec{\alpha}_n = \half \sum_{n=-\infty}^{-1} \left\{\vec{\alpha}_n \cdot \vec{\alpha}_{-n} + \eta\munu \underbrace{[\alpha^\mu_{-n}, \alpha^\nu_n]}_{-n \eta^{\mu \nu}}		\right\} + \half \sum_{n=0}^\infty \vec{\alpha}_{-n} \cdot  \vec{\alpha}_n \\
	&=\half \sum_{n\in\Z} N\left(\vec{\alpha}\cdot \vec{\alpha}_n\right) + \frac{d}{2} \sum_{n=1}^\infty n = L^{qu}_0 -a,
\end{align*}
we capture the ambiguity in a divergent \emph{normal ordering constant} fixed by renormalization later.

\begin{mybox}{Virasoro algebra }
	The \emph{Virasoro algebra} formed by the quantum Virasoro generators 
	\be 
	\label{eq:stringVirasoroAlgebra}
	[L_m,L_n]= (m-n) L_{m+n} + \frac{c}{12} m (m^2-1) \delta_{m,-n}
	\ee 
	is a central extension by $\mC$ of the classical Witt algebra $\{L_m,L_n\}_{PB} = (m-n)L_{m+n}$ satisfied by the classical Virasoro generators. The \emph{central charge} $c = \eta^\mu_\mu = d$ is given by the number of scalars $X^\mu$. The fact that $c \neq 0$ indicates a quantum analogy of the worldsheet's conformal symmetry.
\end{mybox}
To exclude negative norm states from the physical Hilbert space and ensure a unitary theory, we impose (with Ehrenfest's theorem in mind) they \emph{physical state condition} 
\be 
\label{eq:stringPhysicalStatecond}
(L_m-a \delta_{m,0}) \ket{\phi} = 0 \quad \forall m \geq 0 \text{ and } \forall \ket{\phi}\in \mH_{phys}.
\ee 
Since the (quantum) mass shell condition arises from the level-zero Virasoro constraint, the normal ordering constant $a$ affects the string mass. The structure of the physical Hilbert space is a tower of string excitations with increasing mass according to the number of excitations counted by $N$:
\be
M^2_{op} \ket{\phi} = \left(\frac{1}{\alpha^\prime} (N-\alpha) + T^2 \Delta \vec{x}^2\right)\ket{\phi}.
\ee 
$T^2 \nabla \vec{x}^2$ is the energy contribution from the string's tension, non-zero only for states stretched between non-coincident D-branes.The closed string states with $M^2_{cl} \ket{\phi} = \frac{2}{\alpha^\prime} (N^++N^--a) \ket{\phi}$ are oganized by the \emph{level-matching condition}
\bse 
(N^+-N^-) \ket{\phi}=0.
\ese 
\begin{enumerate}
	\item For $a>0$, the vacuum $\ket{0,\vec{p}}$ of bosonic string theory is \emph{tachyonic} with $M^2=-\frac{a}{\alpha^\prime}$. This is not inconsistent, but signals an instability of the (naive) vacuum. Such a theory rapidly decays.
	\item Analysis of the level-zero Virasoro constraint on a first-excited level state
	\bse 
	\ket{\phi} = \xi_\mu \alpha^\mu_{-1}\ket{0,\vec{p}}
	\ese 
	reveals
	\bse 
	(L_0-a)\ket{\phi} = \left(\frac{\vec{\alpha}^2_0}{2}+\vec{\alpha}\cdot \vec{\alpha}_{-1} - a\right)\ket{\phi} = \left(\alpha^\prime \vec{p}^2 +1 -a\right) \ket{\phi}\stackrel{!}{0}
	\ese 
	which implies
	\be
	\vec{p}^2= \frac{a-1}{\alpha^\prime}.
	\ee 
	The level-one constrain evaluates to the requirement of transverse polarization $\vec{\xi}$,
	\bse 
	L_1 \ket{\phi}= \half \left(\dots+ \vec{\alpha}_1 \cdot \vec{\alpha}_0+ \vec{\alpha}_0\cdot \vec{\alpha}_1+\dots\right) \ket{\phi} = \sqrt{2 \alpha^\prime} \vec{p}\cdot \vec{\xi} \ket{\phi} \stackrel{!}{=} 0
	\ese 
	which implies that
	\be 
	\vec{p} \cdot \vec{\xi} =0.
	\ee 
	All higher constraints are vacuous (automatically satisfied). Since for $a>1$ we have $\vec{p}^2>0$, we can choose $\vec{p}$ such that $p^0=0$. Then a purely $\xi^0$-polarized state fulfils $\vec{p}\cdot \vec{\xi}=0$, but, due to 
	\bse 
	\braket{\phi}{\phi} = \bra{0,\vec{p}} (\xi_\mu \alpha^\mu_{-1})^\dagger (\xi_\mu \alpha^\mu_{-1}) \ket{0,\vec{p}} = \xi^\mu \xi_\mu,
	\ese 
	has negative norm for every $\xi^0>0$. Thus $a\leq 1$ is necessary for a unitary quantum theory. For $\ket{\phi}$ twice excited, we similarly find we need $d \leq 26$.
\end{enumerate}

\subsection{Lightcone quantization}
It is convenient in this procedure to introduce lightcone coordinates also for spacetime
\bse 
X^\pm = \frac{1}{\sqrt{2}} (X^0 \pm X^{d-1}), \; X^i, \; i\in \{1,\dots,d-2 \},\; \eta_{\pm \mp} =-1=\eta_{\mp\pm} , \; \eta_{ij}=\delta_{ij},
\ese 
so that $\vec{X}\cdot \vec{X}= -2 X^+ X^- + \sum_i (X^i)^2=- 2 X^+ X^- + \vec{X}^2_\perp$.\\
The key idea of lightcone quantization is to use the infinite dimensional residual symmetry generated by the conformal KIlling vectors fulfilling \ref{eq:stringConformalKilling} to gauge away an infinite number of oscillator derees of freedom, i.e. we set $\alpha^+_n=0 \forall n \neq 0$. This is possible because $\tau = \half (\xi^++\xi^-)$ fulfills the string field's e.o.m. $\partial_+ \partial_- \tau =0$. We can thus find a conformal Killing transformation that reshapes the worldsheet so that its time-axis agrees with one of the spacetime coordinates, say $X^+$, i.e. 
\bse 
X^+ = \frac{2 \pi \alpha^\prime}{l} p^+ \tau + x^+
\ese 
(which is just the mode expansion with all modes except $\alpha^+_0$ set to zero).\\
Of course, this procedure generally breaks Lorentz covariance as it singles out one coordinate!\\
But it enables solving the Virasoro constraints at the classical level. By \ref{eq:stringPactionEMtensorLightcone}, $T_{ab}\stackrel{!}{=}0$ becomes
\bse 
-2 (\partial_\tau \vec{X}\pm \partial_\sigma \vec{X})^+ (\partial_\tau \vec{X} \pm \partial_\sigma \vec{X})^- + (\partial_\tau \vec{X} \pm \partial_\sigma \vec{X})^2_\perp \stackrel{!}{=}0.
\ese 
Inserting the string field expansions turns the Virasoro constraints into an interdependence of modes
\be
\label{eq:stringModesIndependent} 
\alpha^-_n = \frac{1}{\sqrt{2\alpha^\prime} p^+} \half \sum_{i=1}^{d-2} \sum_{m\in\Z} \alpha^i_{n-m} \alpha^i_m.
\ee
Inserting spacetime lightcone coordinates into the flat-gauge Polyakov action from \ref{eq:stringPactionflatgauge} yields
\be 
S_P = \frac{T}{2} \int_\Sigma \md^2 \xi \left[( \partial_\tau \vec{X})^2_\perp - (\partial_\sigma \vec{X})^2_\perp\right] - \int_{-\infty}^{\infty} \md \tau p^+ \partial_\tau q^-.
\ee 
Following the standard quantization procedure gives canonically conjugate variables $X^+ \leftrightarrow \Pi^i$ and $p^+ \leftrightarrow \partial_\tau q^-$ with $q^-= \frac{1}{l} \int_0^l \md \sigma X^-$. Eq. \ref{eq:stringModesIndependent} and $L_m$ quantize as $\alpha^i_{n-m} \alpha^i_m\rightarrow N(\alpha^i_{n-m} \alpha^i_m) - a \delta_{m,0}$.
\\
Since the Virasoro constraints are implemented explicitly, all excitations created by transverse modes $\alpha^i_{-m}$ are automatically physical and the spectrum is manifestly free of ghosts.
\begin{mybox}{}
	\emph{Criticality} in lightcone quantization follows from requiring Lorentz covariance. A long calculation reveals that the \emph{Lorentz algebra} is non-anomalous only if $d=26,a=1$.
\end{mybox}
The quantized Hamiltonian for open NN strings is $H=\frac{\pi}{l}(L_0-a)$. It needs to be \emph{renormalized} due to the divergent $a$. First, we regularize with a cutoff $\Lambda$:
\bse
a= \frac{d-2}{2} \sum_{n=1}^{\infty} n = \lim_{\Lambda \rightarrow \infty} \frac{d-2}{2} \sum_{n=1}^{\infty} n \left(e^{- \frac{\pi}{l \Lambda}} \right)^n.
\ese 
Using $\sum_{n=1}^{\infty} n q^n= q \frac{\md}{\md q} \sum_{n=1}^{\infty} q^n= \frac{q}{(1-q)^2}$, this becomes
\be 
\frac{\pi}{l} a = \lim_{\Lambda \rightarrow \infty} \frac{\pi}{l} \frac{d-2}{2} \frac{e^{-\frac{\pi}{l\Lambda}} }{(1-e^{-\frac{\pi}{l \Lambda}} )^2} = \lim_{\Lambda \rightarrow \infty} \frac{d-2}{2} \left(\frac{l}{\pi} \Lambda^2 - \frac{\pi}{l}\frac{1}{12} +  \mO(\Lambda^{-1}) \right).
\ee 
\begin{enumerate}
	\item The divergent $\Lambda^2$-term scales with $l$. It can be absorbed by adding (via renormalization) a cosmological constant counterterm 
	\bse 
	S_{cc} \propto \Lambda^2 \int_\Sigma \md^2 \xi \sqrt{-h}
	\ese 
	to the bare Polyakov action.\footnote{We have to cancel the divergent term entirely to preserve conformal invariance: A non-zero cosmological constant term would break conformal symmetry already at the classical level. $a\neq 0$ also breaks conformal invariance, but only in the form of an acceptable anomaly at the quantum level.}
	\item The finite term is only present due to the finite size of the string (it disappears for $l\rightarrow \infty$). There exists no local counterterm that could be added to absorb it. This term is therefore physical and define the \emph{Casimir energy} of the string as 
	\bse 
	\frac{\pi}{l} a = \frac{\pi}{l} \frac{d-2}{24}.
	\ese 
\end{enumerate}
For mixed rather than pure NN boundary conditions, the normal ordering constant increases by $\frac{1}{24}$ per $NN-/DD$-dimension and decreases by $-\frac{1}{48}$ per $ND-/DN$-dimension. Thus
\bse 
a_{tot} = \frac{d-2}{24} - \frac{n_{ND}+n_{DN}}{16},
\ese 

\subsection{String spectrum}
\subsubsection{Little group}
In a $d$-dimensional Lorentz covariant theory, states form irreducible representations of the subgroup $\mathbb{S}$ - called \emph{little group} or \emph{stabilizer} - of the $d$-dimensional Poincaré group $SO(1,d-1)\rtimes \mR^{1,d-1}$\footnote{Note that we used the semi-direct product here to construct the Poincaré group.} that leave their momentum $p^\mu$ invariant. Depending on whether $p^\mu$ is space-/light-/timelike, $\mathbb{S}$ is 
\begin{enumerate}
	\item For $\vec{p}^2>0$, we can Lorentz transform to get $\vec{p}=(0,p,0,\dots,0)$ and hence $\mathbb{S}=SO(1,d-2)$. An example is the tachyonic ground state $\ket{0,\vec{p}}$ with $\vec{p}^2= \frac{a}{\alpha^\prime}$, a scalar of the little group $SO(1,24)$.
	\item For $\vec{p}^2=0$ we can rotate coordinates so that $\vec{p}=(p,p,0,\dots,0)$, i.e. $\mathbb{S}=SO(d-2)$. States of this type are the first-level massless transverse excitations $\xi_i \alpha^i_{-1}\ket{0,\vec{p}}$. They are spacetime vectors in the fundamental representation $\yng(1)$ of $\mathbb{S}=SO(24)$.
	\item $\vec{p}^2<0$ admits $\vec{p}=(p,0,\dots,0)$, i.e. $\mathbb{S}=SO(d-1)$. All massive states (i.e. at second excited level or higher) form irreps of $\mathbb{S}=SO(25)$. E.g. a second-level state of the form $(\xi_i \alpha^i_{-2}+\zeta_{ij} \alpha^i_{-1} \alpha^j_{-1})\ket{0,\vec{p}}$ has $24+\frac{24}{2}(24+1)=324$ polarization degrees of freedom which combine into the symmetric traceless representation $\yng(2)$ of $SO(25)$.
\end{enumerate}
\subsubsection{$Dp$-brane}
A \emph{D$p$-brane} is a $(p+1)$-dimensional hypersurface on which open strings can end, fixing them in place in the dimensions normal to it. DD boundary conditions allow for momentum flow off the string ends which implies that $D$-branes are dynamical (albeit non-perturbative) objects themselves.
\subsubsection{Open string spectrum}
\begin{enumerate}
	\item The presence of a single $Dp$-brane allows the following low-level states\footnote{The lightcone coordinates $X^\pm=\frac{1}{\sqrt{2}}(X^0 \pm X^{25}) $ mus lie in $NN$ dimensions for a treatment within lightcone quantization. }:
	\begin{enumerate}
		\item The ground state $\ket{0,\vec{p}}$ can have nonzero momentum $\vec{p}$ only in $NN$ dimensions along the brane.
		\item Excitations $\xi_i \alpha^i_{-1}\ket{0,\vec{p}}$ \emph{parallel} to the brane for $i\in \{1,\dots,p-1\}$ form a massless vector from the perspective of the $Dp$-brane. Its interactions identify it as a gauge potential. Thus a \textbf{single brane hosts a }$U(1)$ \textbf{ gauge theory !} Excitations \emph{normal} to the brane $\xi_a \alpha^a_{-1} \ket{0,\vec{p}}$ for $a\in \{p,\dots,24 \}$ form $24-p$ massless scalars. They are the \textbf{Goldstone bosons} associated with spontaneous breaking of the $26$-dimensional Poincaré symmetry by the brane.
	\end{enumerate}
\item Strings stretched between parallel $Dp$-branes located at $x^a_1$ and $x^a_2$ receive a mass contribution $T^2 \sum_a (x^a_2-x^a_1)^2$ from tension, rendering the above gauge and Goldstone excitations massive.
\item For $N$ coincident $Dp$-branes, states need to be labelled by \emph{Chan-Paton factors} $r,s\in \{1,\dots,N\}$ enumerating the branches to keep track of the boundaries. We get the same states listed under $1$, but $N^2$ copies of each, i.e. $N^2$ massless vectors and $N^2(24-p)$ massless scalars. The vectors enhance the original $U(1)$ gauge symmetry to a non-Abelian $U(N)$. States can be expanded in terms of the $N^2$ hermitian $N\times N$-Chan-Paton -matrices $\mathbf{\lambda}^a$ that span the Lie algebra of $U(N)$.\footnote{In orientifolded theories, also $SO(N)$ and the symplectic $Sp(2N)$ are possible gauge groups of coincident branes.}
\end{enumerate}
\subsubsection{Closed spring spectrum}
The polarization two-tensor $\xi_{ij}$ of the first excited level decomposes into irreps of the little group $SO(24)$: $\xi_{ij}=g_{ij}+B_{ij}+\phi \delta_{ij}$, where $g_{ij}$ is symmetric traceless and describes massless, transversely polarized spin $2$ particles, i.e. \emph{gravitons}, while the antisymmetric $B_{ij}$ is called \emph{Kalb-Ramond tensor field} and models a generalized (i.e. higher-rank) gauge potential. Lastly, $\phi$, the trace part of $\xi_{ij}$, is a scalar field called the \emph{dilaton}.

\subsection{Covariant quantization}
The modern covariant approach to quantization utilizes the path integral which is particularly suitable for theories with gauge symmetries, and a powerful tool to compute string interactions.\\
The naive partition function $Z=\int \mD X \mD h e^{iS_p[X,h]}$ overcounts due to gauge-equivalent configurations of the auxiliary field $h_{ab}$. The solution is to isolate the integral over gauge space, in this case over all diffeomorphisms $\epsilon_a$ and Weyl rescalings $\Lambda$, and cancel it by division with the gauge group's volume. This can be achieved with the Faddeev-Popov gauge fixing procedure and yields
\be 
\label{eq:stringQuantGeneratingFctl1}
Z=\int \mD X \det(\mathbf{P}) e^{i S_P[X,\hat{h}]} 
\ee 
with $\hat{h}_{ab}$ an arbitrary reference metric, and $\det(\mathbf{P})$ the \emph{Faddeev-Popov determinant}, stemming from the Jacobian of the transformation used to factor out $\mD \epsilon$ and $\mD \Lambda$ and cancel with $\text{Vol}^{-1}_{\text{diff }\times \text{Weyl}}$,
\be 
\mD h \rightarrow \mD \epsilon \mD \Lambda \det \left(\frac{\partial(\mathbf{P}\mathbf{\epsilon}, \Lambda)}{\partial (\mathbf{\epsilon},\Lambda)} \right) = \mD \epsilon \mD \Lambda \det \begin{pmatrix}
	\mathbf{P}&0 \\
	0& 1\\
\end{pmatrix}.
\ee 
We assumed every metric $h_{ab}$ can be transformed to $\hat{h}_{ab}$ for precisely one combination of $\epsilon_a$ and $\Lambda$. There is, however, a double mismatch:
\begin{enumerate}
\item The conformal Killing vectors generate as yet unfixed residual gauge transformations which leave the metric invariant and must not be integrated over to avoid overcounting.
\item For worldsheets of complicated topology, the metric contains global properties - the moduli - not accounted by local gauge transformations. For topologies more complicated than the vacuum, we must sum over these moduli by hand.
\end{enumerate}
\subsubsection{Characteristic of the ghosts}
By introducing the Grassmann-valued Faddeev-Popov ghost $c^a(\xi)$ and antighost $b_{ab}(\xi)$, $\det(\mathbf{P})$ can be expressed as the Berezin integral 
\bse 
\det(\mathbf{P}) = \int \mD b \mD c \exp \left[\frac{1}{4 \pi} \int_\Sigma \md^2 \xi \sqrt{-\hat{ h}} \mathbf{b}\cdot (\mathbf{P}\cdot c)\right]
\ese 
with which \ref{eq:stringQuantGeneratingFctl1} becomes
\be 
Z=\int \mD X \mD b \mD c e^{i(S_P+S_g)}.
\ee 
$c^a(\xi)$ and $b_{ab}(\xi)$ are anti-commuting, fermionic fields with integer spin in violation of the spin-statistics theorem, thus producing negative norm states. They are governed by the ghost action
\be
\label{eq:stringQuantGhostaction}
S_g = \frac{-i}{2 \pi}  \int_\Sigma \md^2 \xi \sqrt{-\hat{h}} \hat{h}^{ab} c^d \nabla_a b_{bd} \stackrel{lcg}{=} \frac{i}{\pi} \int_\Sigma \md^2 \xi (c^+ \partial_- b_{++}+c^- \partial_+ b_{--})
\ee 
and the equations of motion $\nabla^a b_{ab} = 0=\partial_\mp b_{\pm \pm}$ and $P^c_{ab} c_c=0=\partial_\mp c^\pm$ imply $c^a$. The latter tells us that the ghost vectors $c^a$ are in one-to-one correspondence with the conformal Killing vectors $ \epsilon_a$.\\
\\
Ghosts and antighosts are canonically conjugate fields. Their modes $b_n,c_n$ fulfill the anti-commutaion relations
\bse 
\{c_m,b_n\} = \delta_{m,-n} \quad \{c_m,c_n\} = \{b_m,b_n\}=0.
\ese 
In terms of the ghost modes, the ghost Virasoro generators read
\bse 
L^g_m = \sum_{n\in\Z} (m-n) N(b_{m+n} c_{-n}).
\ese 
the $L^g_m$ satisfy the \emph{ghost Virasoro algebra}
\be 
\label{eq:stringGhostVirasoro}
[L^g_m, L^g_n] = (m-n) L^g_{m+n} + \frac{m}{6} (1-13 m^2) \delta_{m,-n}.
\ee 
Defining $L^{tot}_m=L^X_m+L^g_m-a^{tot} \delta_{m,-n}$ yields the combined \emph{ghost and bosonic Virasoro algebra} 
\be 
\label{eq:stringGhostBosonicVirasoro}
[L^{tot}_m,L^{tot}_n]= (m-n) L^{tot}_{m+n} + \left[\frac{c^{tot}}{12} m(m^2-1) + 2 m(a^{tot}-1)\right]\delta_{m,-n},
\ee 
with the central charge $c^{tot}=c^X+c^g$ where $c^g=-26$, $c^X=d$ in $\mR^{1,\md-1}$, and total normal ordering constant $a^{tot}=a^X+a^g$, where $a^g=+ \frac{1}{12}$ and $a^X=\frac{\md}{24}$ in a covariant gauge that treats $X^\pm$ as independent d.o.f. The presence of a central term signals a Weyl anomaly of the full action $S_p+S_g$ and, hence, the path integral. But we used Weyl invariance to factor out the integration over gauge-equivalent metrics. Self-consistency thus requires that the central term vanishes which is the case precisely in $\md=26$ (this really fixes the central charge $c^X$ and only indirectly constrains $\md$).


\subsubsection{BRST symmetry}
In path integral quantization of gauge theories, the physical state condition is implemented via the \emph{BRST symmetry}, a global, fermionic, residual symmetry of the full action $S_p + S_g$ invariant under
\be 
\label{eq:stringBRST}
\delta_\epsilon X^\mu = \epsilon(c^+ \partial_+ + c^- \partial_-) X^\mu, \; \delta_\epsilon c^\pm = \epsilon (c^+ \partial_++c^- \partial_-) c^\pm, \; \delta_\epsilon b_{\pm \pm} = i \epsilon (T^X_{\pm \pm} + X^g_{\pm \pm}),
\ee 
where $\epsilon$ is a global Grassmann-valued parameter. BRST invariance is present even after gauge fixing $h_{ab}=\eta_{ab}$. It is generated by the nilpotent, Hermitian, conserved charge $Q_B$ via the brackets
\be 
\delta_\epsilon X^\mu = \epsilon [Q_B,X^\mu], \; \delta_\epsilon c^\pm = \epsilon \{Q_B,c^\pm\}, \; \delta_\epsilon b_{\pm \pm} = \{Q_B,b_{\pm \pm} \}.
\ee 
A long calculation reveals that nilpotence $Q^2_B = \half \{Q_B,Q_B\}\stackrel{!}{=}0$, as required for consistency of the BRST symmetry, is equivalent to absence of the Weyl anomaly, i.e. zero central extension in \ref{eq:stringGhostBosonicVirasoro}. \\
\\
A physical state must be gauge invariant. Since $Q_B$ acts on $X^\mu$ like the residual gauge transformations generated by conformal Killing vectors, a physical state must be invariant under a BRST transformation, i.e.
\bse 
Q_B \ket{\phi}=0 \quad \forall \ket{\phi}\in\mH_{phys}.
\ese 
This is not sufficient, however. As in quantization of YM theory (compare \ref{subsec:BRSTYM}), we find that due to nilpotence, all states in $\mH$ lie either in the kernel $\ker(Q_B)$ or image $\text{Im}(Q_B)$ of $Q_B$. The latter states are null, i.e. orthogonal to all states including themselves. The physical (positive-norm) Hilbert space is given by the cohomology of $Q_B$, i.e.
\be 
\label{eq:stringBRSTHilbert}
\mH_{phys}= \mathcal{C}(Q_B) = \frac{\ker(Q_B)}{\text{Im}(Q_B)}. 
\ee 
States differing only by elements of $\text{Im}(Q_B)$ are in the same equivalence class and transform into one another by gauge transformation.






\section{String interactions}
A key property of string 

































